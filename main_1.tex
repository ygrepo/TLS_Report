%The mandatory template presets
\PassOptionsToPackage{table,xcdraw}{xcolor}
\documentclass{article}
\usepackage{fullpage}
\usepackage{graphicx}
\usepackage{booktabs}
\usepackage{amsmath}
\usepackage{amsthm}
\usepackage{parskip}
\usepackage{hyperref}
\usepackage{pdfpages}
\usepackage[table,xcdraw]{xcolor}
\usepackage{pdflscape}
\usepackage{array}

%NEW - Use british english.
\usepackage[british]{babel}

%NEW - allow use of N,C,Q,R, etc..
\usepackage{amsfonts}



%NEW - allows for use of theorems, definitions, and so on.
\newtheorem{theorem}{Theorem}
\newtheorem{lemma}{Lemma}
\newtheorem{corollary}{Corollary}
\newtheorem{definition}{Definition}[section]
\newtheorem{comment}{Comment}
\newtheorem{conjecture}{Conjecture}


\title{TLS report}

\author{Smurity Karale, Yves Greatti}
\date{May 2025}

\begin{document}

\maketitle

Whilst Machine Learning (ML) techniques are used within the considered literature, the vast majority of papers use them only for the purposes of statistical analysis. This, in turn, is done in aid of identifying predictors of TLS and TLS-related complications. The exceptions to this are \cite{mato2004predictive}, 
\cite{montesinos2008tumor}, and  \cite{xiao2024prediction}. Our analysis begins by exploring \cite{mato2004predictive}, 
\cite{montesinos2008tumor}, and  \cite{xiao2024prediction} individually, and then moves into exploring how ML techniques have been used to perform statistical analysis to locate predictors of TLS.

\section{ML for prediction of TLS}

\cite{mato2004predictive} Took data from 194 patients between the ages of 18 and 86 (of which 19 also had TLS) with Acute Myelogenous Leukemia (AML) and performed statistical analysis on this data in order to find predictors of Tumor Lysis Syndrome (TLS). The authors performed both univariate and multivariate analysis, which were done via logistic regression, a binary classification model, which works by fitting a logistic curve that splits the data into two user-defined categories, in this case, patients that have TLS and patients that do not have TLS. By looking at how important specific variables are to the model’s predictions, one can infer the likelihood that a variable is a predictor of TLS. In this case, the authors found that uric acid “(UA) (p=.0003), Cr (p=.0025), [lactate dehydrogenase] LDH (p=.0001), [White Blood Cell count] WBC (p =.0058), gender (p =.0064), and [Chronic Myelomonocytic Leukemia] CMML (p=.0292)” were predictors of TLS. When the authors used multivariate analysis, LDH “(p=0.01, OR 3.01, 95\% CI 1.5–6.2) and UA p =0.01, OR 2.00, 95\% CI 1.4–2.8)” were predictors. While these values are statistically significant, the authors did not mention effect size.

There is some debate on how much data is required for logistic regression, a general rule of thumb is given by:

\[\frac{10C}{P}\]

Where C is the number of covariates (potential predictors of TLS in this case) and P is the smallest proportion of negative or positive cases. It is not stated precisely how many predictors were tested at once during multivariate analysis, but as 6 were found, we can assume \(C \geq 6\). Hence:

\[P = \frac{19}{194} \approx 0.1 \]
\[\frac{10C}{P} \approx \frac{10 \times 6}{0.1} = 600\]

The study had 194 datapoints and used less data than might be recommended for their multivariate logistic regression analysis. When using less data than recommended, there is a risk of the results not being representative of the general population; however, in some cases (especially within healthcare), such a situation is not always avoidable. Their univariate logistic regression did not have this issue, as for C = 1, we find that approximately 100 datapoints would be recommendable. As such, there may be precedence for UA, LDH, WBC,  sex, and CMML to be predictors of TLS.

After using statistical analysis to find predictors of TLS, the authors used LDH and UA to make a points-based predictive model of TLS (PPS-TLS score). Points were assigned based on LDH and UA values, with a bias given to UA. The PPS‑TLS score varies between 0 and 6, and is challenging to use it as a binary classifier: if you set a low threshold (e.g., score \(\ge 2\) or 3), you will catch all the TLS cases but you’ll have so many positives that the test doesn’t significantly change the pre‑test odds \((\mathrm{LR}+ \sim1 \text{--}1.7)\). The median of the patients with TLS was 5, with a sensitivity of 0.63 and a specificity of 0.85, showing that the score has modest discriminative ability. 
A more rigorous approach would be to plot the whole ROC curve (and report the AUC), choose the threshold that maximizes Youden’s \(J\) index, and then assess calibration or validate that cut‑off in an independent sample.



Whilst it is easy to interpret a system wherein predictors are assigned points, the accuracy of the model and the fact that it was based on only two predictors found via multivariate analysis on a small dataset means that it is unlikely to be helpful in a clinical setting. The authors also state that the model has not been externally validated. 

\cite{montesinos2008tumor} took 772 adult patients, of which 130 had TLS. The authors began by making all of their continuous variables into categorical intervals before doing a univariate analysis to find important variables, which was accomplished via a chi-squared test. Afterwards, the statistically significant variables were used to build a stepwise logistic regression model. Using our rule of thumb from earlier:
\[P =\frac{130}{772} \approx 0.17 \]
\[ \frac{10C}{P} \approx \frac{10\times9}{0.17} \approx 529 \]
Hence, the authors likely had enough data to create such a model. From their multivariate analysis, the authors found that WBC, UA, and LDH were predictors of TLS.

Finally, the authors created a scoring system based on their found predictors. This scoring system was a points-based system similar to the one created by \cite{mato2004predictive}. Whilst it is easy to interpret a system wherein predictors are assigned points, the accuracy of the model means that it is unlikely to be a good diagnostic tool on its own, and instead might be more useful as part of a holistic diagnostic system that uses multiple different tools to make a diagnosis.

(Yao Xiao, 2024) Took data from 2,243 patients under the age of 18 with ALL (of which 199 had TLS). The authors used this data to train a variety of machine learning models, with the goal of predicting whether a patient has TLS. It is worth noting that the group used a miss-forest model to fill in missing values within the dataset, although this is unlikely to affect the overall results from a machine learning perspective.

To start the study, the authors used LASSO regression to find predictors of TLS. LASSO regression, also known as l1 regression. LASSO regression minimizes the magnitude of coefficients in a Machine Learning model by punishing the model by adding the absolute value of the coefficients multiplied by some coefficient to the loss of the model. At a high level, this incentivizes the model to send the coefficients of unimportant variables to 0. As a result, by performing the LASSO regression, it is possible to tell which variables are not important as they will have been sent to 0 during the LASSO regression. It is worth noting that if two variables are heavily correlated (i.e. age and weight in children) then LASSO regression may choose one to send to 0 even though both can provide value to the model (\textbf{YG: It is important to note that if two variables are highly correlated (e.g., age and weight in children), LASSO regression may shrink one to zero, even if both contribute provide value to the model.}). It is also worth noting that which variables are sent to 0 depend heavily on the training data, and it can be hard to interpret why a specific feature was chosen as important by LASSO regression (\textbf{YG:Additionally, the selection of variables set to zero is highly dependent on the training data, making it challenging to interpret why LASSO prioritizes certain features over others.}). The authors of the paper minimized these disadvantages by using 10-fold cross-validation, however, they are still risks that should be considered when constructing models. The authors found that the LASSO regression selected 'FAB type, WBC, phosphorus, calcium, potassium, UA, AST, blood glucose, occurrence of infection, AKI, cardiac arrhythmia, and type of steroid used in initial 2 induction chemotherapy' as the 12 most important variables. WBC was also found as a predictor of TLS in both (Tony H. Truong, 2007,[3] and Anthony R. Mato, 2004 (\textbf{YG:Reference?}). In addition, Anthony R. Mato (2004) found UA to be a predictor of TLS, although this was in a study performed on adults rather than children, with a different form of cancer.

After using LASSO regression to find the 12 most important variables, the researchers created 4 different models to predict TLS from these variables: CatBoost, logistic regression, random forest, and a Support Vector Machine (SVM). Exploring multiple different models is good practice, as it is often difficult to tell how a specific model structure will perform on a dataset without actually testing it. It is worth mentioning that whilst it is good that they tested multiple different hyperparameter configurations for each model (\textbf{YG: While it is commendable that they tested multiple hyper-parameter configurations for each model to ensure fairness, the authors do not appear to have evaluated the finalized models on multiple different training datasets.}), ensuring no model was unfairly disadvantaged, the authors do not appear to have tested multiple different training
datasets with the finalized models. It is worth noting that model performance can be highly dependent on the training data provided to it, as well as how the model started in that specific run (which is usually random), so not performing repeated tests runs the risk of models doing better/worse purely because of the data that is supplied to them. This is particularly important as the models explored by the authors all performed relatively similarly, and so changes to the training data and running repeated tests may have changed the results (\textbf{YG: It is important to recognize that model performance can be heavily influenced by the training data used and the initial conditions of a given run, which are typically random. Without repeated testing, there is a risk that model performance varies only due to differences in the data provided. This is especially relevant since the models analyzed by the authors performed similarly, and so changes to the training data and running repeated tests may have changed the results.}.

It's also worth mentioning that within the paper, once the 12 predictors of TLS were found the authors did not explore the biological and clinical relevance. Without a strong biological argument, the paper loses some of the impact it could have had (\textbf{YG:It is also important to note that after identifying the 12 predictors of TLS, the authors did not investigate their biological and clinical significance. Without a strong biological argument, the paper loses some of its potential impact.}.

Ultimately, the paper shows mostly good practice within AI, and uses appropriate techniques. Whilst one model (CatBoost) was shown to be better than the others within the paper, this may not always hold in practice. It is also worth noting that whilst the authors suggest that CatBoost “can be incorporated into a clinical decision support system”, CatBoost only had an accuracy of 75.8\(\%\). Whilst in theory, CatBoost is interpretable, in practice it can be difficult to tell why CatBoost has made a specific prediction - and this can limit the practical usability of the model (\textbf{YG:
Although CatBoost outperformed the other models in the paper, this may not always in practice hold true. Additionally, while the authors suggest that CatBoost "can be incorporated into a clinical decision support system," its accuracy was only 75.8\%. While CatBoost predictions are theoretically interpretable, understanding the reasoning behind its predictions in practice can be challenging, which may limit its practical applicability.}.

\section{ML for statistical analysis of TLS factors}

%%%%%
%%%%%
%%%%%
%%%%%

Within the considered literature, the vast majority of authors are focused on finding predictors of TLS and TLS-related complications. The methods used for this are typically well-established statistical techniques, with a heavy emphasis on logistic regression and, to a lesser extent, COX regression. Below, we have provided a table that provides information about each of the papers. Note that \cite{mato2004predictive}, 
\cite{montesinos2008tumor}, and  \cite{xiao2024prediction} were discussed earlier, and are thus excluded from the table.

\begin{landscape}
\begin{table}[]
\hspace*{-3.5cm}\begin{tabular}{>{\raggedright\arraybackslash}p{1cm}|l|l|l|l|l|l|}
\hline
\cellcolor[HTML]{FFFFFF}{\color[HTML]{333333} Name} & 
Model(s) used & 
\begin{tabular}[c]{@{}l@{}}Number of\\ patients\\ in study.
\end{tabular} & 
\begin{tabular}[c]{@{}l@{}}Number of TLS\\ patients in study\end{tabular} & Finding & Notes                                                               \\
\hline
\multicolumn{1}{|l|}{\begin{tabular}[c]{@{}l@{}}Features at presentation predict children\\ with acute lymphoblastic leukemia at low\\ risk for tumor lysis syndrome.\\ (Tony H. Truong, 2007)\end{tabular}}           
& \begin{tabular}[c]{@{}l@{}}Logistic regression\\ (both univariate and\\ multivariate)\end{tabular}
& 74
& 328
& \begin{tabular}[c]{@{}l@{}}“sex, age, WBC, mediastinal mass,\\ hepatomegaly, splenomegaly, and\\ T-cell immunophenotype” were\\ all predictors of TLS.\end{tabular}
& N/A

\\
\hline
\multicolumn{1}{|l|}{\begin{tabular}[c]{@{}l@{}}Serum phosphate level and its kinetic as\\ an early marker of acute kidney injury\\ in tumor lysis syndrome (Marie Lemerle,\\ 2022)\end{tabular}}
& COX regression
& 120
& 120
& \begin{tabular}[c]{@{}l@{}}“increases in serum phosphate and\\ LDH appear to be early and reliable\\ biomarkers of AKI in tumour lysis\\ syndrome”\end{tabular}         
& \begin{tabular}[c]{@{}l@{}}Studied acute kidney injury\\ predominantly, instead of the\\ factors that cause TLS.\end{tabular}

\\
\hline
\multicolumn{1}{|l|}{\begin{tabular}[c]{@{}l@{}}(Fausto A. Rios-Olais, 2024) Tumor Lysis\\ Syndrome Is Associated with Worse\\ Outcomes in Adult Patients with Acute\\ Lymphoblastic Leukemia. (Fausto A. Rios\\ -Olais, 2024)\end{tabular}}
& \begin{tabular}[c]{@{}l@{}}Logistic regression\\ (Multivariate), COX\\ regression.\end{tabular}
& 138
& 42                                   & \begin{tabular}[c]{@{}l@{}}UA, LDH, and Male sex were\\ significant predictors of clinical\\ TLS. LDH and WBC were\\ significant predictors of TLS.\end{tabular}
& N/A

\\
\hline
\multicolumn{1}{|l|}{\begin{tabular}[c]{@{}l@{}}Uric Acid and the Prediction Models of\\ Tumor Lysis Syndrome in AML (A. \\Ahsan Ejaz, 2015)\end{tabular}}
& \begin{tabular}[c]{@{}l@{}}Logistic regression\\ (Multinomial)\end{tabular}
& 183
& \begin{tabular}[c]{@{}l@{}}48 LTLS and\\ 10 CTLS\end{tabular}
& \begin{tabular}[c]{@{}l@{}}WBC is a better predictor of TLS\\ than other predictors.\end{tabular}
& N/A                                                                        
\\
\hline
\multicolumn{1}{|l|}{\begin{tabular}[c]{@{}l@{}}Predictors for Severe Tumor\\ Lysis Syndrome (Scott Wirth, 2012)\end{tabular}}
& Logistic regression
& 1327
& N/A                                  & \begin{tabular}[c]{@{}l@{}}Sex, age, cancer type, risk category,\\  Serum Creatinine levels, Blood\\ Urea Nitrogen levels, Magnesium\\ levels, Phosphorus levels, all\\ predicted severe outcomes.\end{tabular}
& \begin{tabular}[c]{@{}l@{}}Studied the likelihood of a\\ severe outcome based on\\ risk category and other\\ standard predictors.\end{tabular}

\\
\hline
\multicolumn{1}{|l|}{\begin{tabular}[c]{@{}l@{}}Risk factors and development of a\\ predictive score model for tumor lysis\\ syndrome in childhood leukemia: a\\ 10-year experience from a single tertiary\\ hospital in Thailand (Pharsai Prasertsan,\\ 2024)\end{tabular}} & \begin{tabular}[c]{@{}l@{}}Logistic regression\\ (univariate,\\ multivariate)\end{tabular}
& 252
& \begin{tabular}[c]{@{}l@{}}51 with TLS, of\\ which 24 had\\ CTLS\end{tabular}
& \begin{tabular}[c]{@{}l@{}}Age, BMI, Mediastinal mass,\\ WBC, LDH, GFR, AST\\ were all identified as risk factors\\ for TLS.\end{tabular}        & N/A                                  

\\
\hline
\multicolumn{1}{|l|}{\begin{tabular}[c]{@{}l@{}}Risk factors for tumor lysis syndrome in\\patients with chronic lymphocytic\\leukemia treated with the cyclin-\\dependent kinase inhibitor,\\ flavopiridol (K A Blum, 2011)\end{tabular}}
& \begin{tabular}[c]{@{}l@{}}Logistic regression\\(univariate,\\multivariate)\end{tabular}
& 116
& \begin{tabular}[c]{@{}l@{}}53\end{tabular}
& \begin{tabular}[c]{@{}l@{}}The authors found 5 statistically\\significant risk factors for TLS:\\WBC, gender, Bulky\\lymphadenopathy, \(\beta\)2-microglobulin,\\ Albumin.\end{tabular} & N/A       

\\
\hline
\multicolumn{1}{|l|}{\begin{tabular}[c]{@{}l@{}}In‐Hospital Outcomes of Tumor Lysis\\Syndrome: A Population‐Based Study\\Using the National Inpatient Sample\\(Urshila Durani, 2017)\end{tabular}}
& \begin{tabular}[c]{@{}l@{}}Logistic regression,\\ linear regression.\end{tabular}
& 28,370
& \begin{tabular}[c]{@{}l@{}}28,370\end{tabular}
& \begin{tabular}[c]{@{}l@{}}“age, weighted Elixhauser comorbidity\\score, insurance, teaching hospital\\ versus non teaching hospital, and\\cancer type were predictors of in‐\\hospital mortality, whereas sex, race, \\facility, and income were not”\end{tabular} & N/A       

\\
\hline
\end{tabular}

\end{table}

\begin{table}[]
\hspace*{-3.5cm}\begin{tabular}{>{\raggedright\arraybackslash}p{1cm}|l|l|l|l|l|l|}
\hline
\cellcolor[HTML]{FFFFFF}{\color[HTML]{333333} Name} & 
Model(s) used & 
\begin{tabular}[c]{@{}l@{}}Number of\\ patients\\ in study.
\end{tabular} & 
\begin{tabular}[c]{@{}l@{}}Number of TLS\\ patients in study\end{tabular} & Finding & Notes                                                               \\
\hline
\multicolumn{1}{|l|}{\begin{tabular}[c]{@{}l@{}}Predictors of in-Hospital Mortality in\\ Tumor Lysis Syndrome \end{tabular}}           
& \begin{tabular}[c]{@{}l@{}}Logistic regression\\ (univariate and\\ multivariate)\end{tabular}
& 997
& 997
& \begin{tabular}[c]{@{}l@{}}“Independent predictors of in-hospital\\ mortality were cardiac dysrhythmias\\ and sepsis."\end{tabular}
& N/A

\\
\hline
\multicolumn{1}{|l|}{\begin{tabular}[c]{@{}l@{}}Identification of Children with Acute\\ Lymphoblastic Leukemia at Low Risk\\ for Tumor Lysis Syndrome (Bahoush \\GR, 2015)
 \end{tabular}}           
& \begin{tabular}[c]{@{}l@{}}Logistic regression\\ (univariate and\\ multivariate)\end{tabular}
& 160
& 41
& \begin{tabular}[c]{@{}l@{}}“CNS and renal involvement and\\ T-cell immunophenotype are among\\ the strongest predictors of TLS."\end{tabular}
& N/A

\\
\hline
\multicolumn{1}{|l|}{\begin{tabular}[c]{@{}l@{}}Risk-based management strategy and\\ outcomes of tumor lysis syndrome in\\ children with leukemia/lymphoma:\\ Analysis from a resource-limited\\ setting
(Gopakumar, 2018) \end{tabular}}
& \begin{tabular}[c]{@{}l@{}}Logistic regression\\ (Multivariate)\end{tabular}
& 224
& 41
& \begin{tabular}[c]{@{}l@{}}“Maintaining hydration and\\ establishing adequate urine output\\ prior to chemotherapy along with the\\ judicious use of rasburicase helps in\\ managing TLS in an economically\\ viable manner in a resource-limited \\setting.”\end{tabular}
& N/A

\\
\hline
\multicolumn{1}{|l|}{\begin{tabular}[c]{@{}l@{}}Plasma uric acid response to rasburicase:\\ early marker for acute kidney injury in\\ tumor lysis syndrome? (Canet, 2014)
 \end{tabular}}
& \begin{tabular}[c]{@{}l@{}}Logistic regression\end{tabular}
& 60
& \(\approx\) 40 (unspecified)
& \begin{tabular}[c]{@{}l@{}}“Baseline plasma uric acid was higher\\ in the AKI group.” and "Conceivably,\\ a smaller response to rasburicase may\\ indicate subclinical AKI at presentation,\\ with a decrease in uric acid clearance."\end{tabular}
& N/A

\\
\hline
\end{tabular}

\end{table}
\end{landscape}

[DO WE WANT TO MENTION THE RESULTS OF THE PAPERS OR LEAVE THAT FOR THE DOCTORS?]

The most common issue within the papers considered is the small datasets that are considered. Many of the papers considered do not have enough data to make reliable claims using logistic regression or more traditional statistical testing, and as such the papers run the risk of having results that cannot be reproduced. It is worth noting that this problem is wide-spread within the medical field due to the ethical and logistical issues with data collection.

The most common approach taken in the papers considered involved using logistic regression, possibly with some pre-processing, to find potential predictors of TLS. Such an approach is well-founded from a mathematical standpoint, however there are many assumptions that are made when using logistic regression. First, logistic regression is limited in how complex the relationships between variables can be, so more complex relationships between variables may be missed. Logistic regression is also sensitive to outliers, and given the small sample sizes used as well as the sometimes messy nature of dealing with patients (\textbf{YG: I will reformulate this part; something like ...as well as dealing with complex real-world datasets}), it is not unreasonable to expect outliers to be present in the data, nor is it unreasonable to assume that such outliers might be difficult to reliably detect. The usage of logistic regression for univariate and multivariate statistical testing, which is common in the papers considered, is not always a good approach (especially in the univariate case), as the models created are typically harder to interpret and reproduce, and often are less robust and reliable for statistical testing than more traditional statistical tests. The decision to create models capable of making predictions about TLS outcomes and then not evaluating them on their ability to do so is also questionable - especially when the main purpose of these models is to do so.

Some papers also explored the usage of COX regression to find how variables may affect the time for important events to occur. COX regression, similarly to logistic regression, is also sensitive to outliers. COX regression also assumes that the effect a variable has is constant over time. As a toy example, COX regression may be able to tell you that people who smoke are more likely to develop lung cancer, however COX regression may also tell you that someone who has been smoking for 5 years is just as likely to develop cancer as someone who started yesterday. Whilst COX regression is well-established, authors typically did not acknowledge or account for these drawbacks in their analysis - which could make results unreliable, and miss potentially useful details.

In some of the papers considered, researchers also explored models that predicted the likelihood of TLS occurring based on provided variables. These approaches were typically points-based systems, wherein variables were assigned “points” based on their value. The points would then be added up, and based on the total, the risk of a specific outcome could be calculated. This approach is inherently well-suited for clinical environments, as it is extremely easy to both interpret the end result, and see the biggest contributing factors to the end result. Unfortunately, such models generally were not accurate enough to be clinically useful as a standalone tool, with it being recommended instead that the models were used as part of a larger diagnosis procedure. (Yao Xiao, 2024) compared a range of well-established, but far less interpretable, AI techniques and found that CATboost had the best performance for predicting TLS, however this approach was still unlikely to be accurate enough to be clinically useful as a standalone tool.

Whilst the methods used in the papers considered are in theory explainable, it is worth noting that it may still be difficult in practice to tell why a model made a specific connection between variables - especially if a large amount of variables are present. This being said, the final result of the models (i.e. tumour size may be a predictor of TLS) is understandable and in a worst case scenario can be verified experimentally after the analysis has taken place. There is also an argument to be made that any analysis which includes large numbers of variables will inherently be difficult to understand. In practice, it should be understood that these models are complex, can find surprising links, but can also sometimes find false relationships. Performing reality checks on the results, as well as potential follow-up studies and analysis, mean that the results from these models can be valuable for advancing TLS research.

It is worth mentioning that it is difficult to tell how well the models created will perform in real-world clinical settings. The accuracy of the models is often not checked, and the data  not released publicly, making it hard to validate what has been done. The papers also  do not create more than one model, or use multiple randomised test/train sets, which is standard within the Machine Learning field to test reliability. The small datasets used mean that models may not perform well for the general population, and the models have major risks of unexpected behaviour in edge cases, extreme cases, and unusual cases.

In the future, the field may want to consider how more sophisticated points-based TLS prediction models could be developed, which may be particularly well suited to clinical settings due to being highly interpretable for both specialists and non-specialists. If this turns out to be unfeasible, the usage of ensemble learning (the usage of multiple different models to make predictions) and transfer learning (taking a big, generalist model and training it to work well in your specific domain) have both been shown to perform well in the sort of low-data situations that appear to be common within the field.


\newpage
\bibliographystyle{plain}
\bibliography{bibliography}

\end{document}

